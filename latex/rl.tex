\documentclass{beamer}

% latex commands and macros

% common sets
\newcommand{\R}{\mathbb{R}} % the set of real numbers
\newcommand{\C}{\mathbb{C}} % the set of complex numbers
\newcommand{\Z}{\mathbb{Z}} % the set of integers
\newcommand{\N}{\mathbb{N}} % the set of natural numbers
\newcommand{\Q}{\mathbb{Q}} % the set of rational numbers
\newcommand{\F}{\mathbb{F}} % usually stands for field
\newcommand{\D}{\mathbb{D}} % the unit disc in the complex plane
\newcommand{\B}{\mathbb{B}} % the unit ball
\newcommand{\es}{\mathbb{S}} % sphere
\newcommand{\T}{\mathbb{T}} % torus
\newcommand{\RP}{\mathbb{R}P} % projective plane
\newcommand{\CP}{\mathbb{C}P} % complex projective plane
\DeclareMathOperator \Gr {Gr} % grassmannian
\newcommand{\ham}{\mathbb{H}} % quaternions

% common operations and connectors
\newcommand{\inner}[2]{\left<{#1},\,{#2}\right>} % inner product
\newcommand{\coset}[1]{\left<{#1}\right>} % groups or ideal generated by some set
\newcommand{\norm}[1]{\left\|{#1}\right\|} % norm of a vector
\newcommand{\abs}[1]{\left|{#1}\right|} % absolute value
\newcommand{\sets}[1]{\left\{{#1}\right\}} % a simple set with no conditions
\newcommand{\set}[2]{\sets{{#1}:\,{#2}}} % a set with conditions
\newcommand{\prn}[1]{\left({#1}\right)} % parentheses 
\newcommand{\brak}[2]{\left[{#1},\,{#2}\right]} % bracket product
\newcommand{\pbrak}[2]{\left\{{#1},{#2}\right\}} % Poisson bracket product

\DeclareMathOperator \csum {\#} % connected sum

\DeclareMathOperator \pp {\,a.e.} % almost everywhere
\DeclareMathOperator \sgn {sgn} % signum function
\DeclareMathOperator \spn {span} % span
\DeclareMathOperator \cspn {\overline{span}} % closed span
\DeclareMathOperator \img {Im} % imaginary part of a complex number
\DeclareMathOperator \rea {Re} % real part of a complex number
\DeclareMathOperator \ind {ind} % index of a function
\DeclareMathOperator \diam {diam} % diameter of a set or a graph
\DeclareMathOperator \res {res} % residue of a complex function
\DeclareMathOperator \dist {dist} % distance function
\DeclareMathOperator \wind {wind} % winding number
\DeclareMathOperator \spectrum {\sigma} % spectrum of an operator
\DeclareMathOperator \spectrumEssential {\sigma_{\text{ess}}} % essential spectrum of an operator

\DeclareMathOperator \smb {smb} % symbol $C^*$-homomorphism

\newcommand{\acton}[2]{\phantom{|}^{{#1}}{{#2}}} % hack to prepend a superscript

\DeclareMathOperator \Syl {Syl} % Sylow subgroup
\DeclareMathOperator \im {im} % image of a function
\DeclareMathOperator \Tor {Tor} % torsion
\DeclareMathOperator \Aut {Aut} % automorphisms
\DeclareMathOperator \End {End} % endomorphisms
\DeclareMathOperator \Mod {Mod} % modules
\DeclareMathOperator \Hol {Hol} % holomorphic functions
\DeclareMathOperator \Sym {Sym} % symmetric group
\DeclareMathOperator \Alt {Alt} % alternating groups
\DeclareMathOperator \Mat {Mat} % matrices
\DeclareMathOperator \Spec {Spec} % spectrum of a ring
\DeclareMathOperator \Max {Max} % maximal spectrum of a ring
\DeclareMathOperator \coker {coker} % cokernel
\DeclareMathOperator \GL {GL} % general linear group
\DeclareMathOperator \SL {SL} % special linear group
\DeclareMathOperator \codim {codim} % codimension of a vector subspace
\DeclareMathOperator \rnk {rnk} % rank of a linear transform
\DeclareMathOperator \Id {Id} % identity function
\DeclareMathOperator \Tr {Tr} % trace of a matrix
\DeclareMathOperator \diag {diag} % diagonal matrix
\DeclareMathOperator \supp {supp} % support of a function

% category theory
\DeclareMathOperator \Ob {Ob} % objects
\DeclareMathOperator \Mor {Mor} % morphism
\DeclareMathOperator \Hom {Hom} % homomorphisms
\DeclareMathOperator \Lin {\mathbf{Lin}} % linear
\DeclareMathOperator \Set {\mathbf{Set}} % sets
\DeclareMathOperator \Vect {\mathbf{Vect}} % vector spaces

% vector calculus
\DeclareMathOperator \diver {div} % divergence
\DeclareMathOperator \grad {grad} % gradient
\DeclareMathOperator \curl {curl} % curl

\DeclareMathOperator \Lie {Lie} % Lie groups
\newcommand{\lie}[1]{\mathfrak{{#1}}} % lie algebra

% operator analysis
\newcommand \slim {\text{s-}\lim} % strong limit
\DeclareMathOperator \Fl{\mathbb{F}l} % spaces of frequency weighted norms
\DeclareMathOperator \Smb{Smb} % symbol operator
\DeclareMathOperator \Alg{Alg} % algebra generated by a set

% statistics
\DeclareMathOperator \Exp{\mathbb{E}} % variance
\DeclareMathOperator \Var{Var} % variance

% miscellaneous

% frequently used abbreviations
\newcommand{\ve}{\varepsilon} % epsilon clearly distinguishable from inclusion
\newcommand{\vf}{\varphi} % more popular way of writing phi
\newcommand{\vn}{\varnothing} % the empty set

% special notation for dissertation
\newcommand{\hprod}{\circ} % Hadamard or entrywise multiplication of matrices
\newcommand{\hs}{{\mathcal{C}_2}} % Hilbert-Schmidt class
\newcommand{\poi}{{\mathcal{P}_1}} % Poincare class
\newcommand{\MellinTransform}{\mathcal{M}} % Mellin Transform
\newcommand{\MellinConvolution}{*_{\MellinTransform}} % Mellin convolution

% environments and fonts for cifar-ten project
\newcommand{\code}[1]{\texttt{{#1}}}

% styles of theorems, definitions and remarks
\theoremstyle{plain}
\newtheorem{thm}{Theorem}%[section]
\newtheorem{lemma}[thm]{Lemma}
\newtheorem{prop}[thm]{Proposition}
\newtheorem{cor}[thm]{Corollary}
\newtheorem{axiom}[thm]{Axiom}
\newtheorem{exercise}[thm]{Exercise}
\newtheorem{claim}[thm]{Claim}
\newtheorem{fact}[thm]{Fact}

\theoremstyle{definition}
\newtheorem{defn}[thm]{Definition}

\theoremstyle{remark}
\newtheorem{remark}[thm]{Remark}
\newtheorem{exmpl}[thm]{Example}
\newtheorem{problem}[thm]{Problem}
\newtheorem{prob}[thm]{Problem}
\newtheorem{fle}[thm]{File}

% page format
\oddsidemargin=0in
\evensidemargin=0in
\textwidth=6in
\topmargin=-0.5in
\textheight=9.5in
\parindent=.375in

% packages
\usepackage{enumerate}
\usepackage{amssymb}
%\usepackage[all,cmtip]{xy}
\usepackage[mathscr]{eucal}
\usepackage{graphicx,color}

\newcommand{\titleChaos}[1]{\title[HW {#1}]{Introduction to Chaos, Homework {#1}}}

%\begin{figure}[ht]
%       \includegraphics[height=1in,keepaspectratio]{03NegativeHelix.jpg}
%       \caption{Negative writhe crossings} \label{fig:negativeWritheCrossings}
%\end{figure}


% packages
\usepackage{beamerthemesplit}
\usepackage{enumerate}
\usepackage{amssymb}
\usepackage[all,cmtip]{xy}
\usepackage[mathscr]{eucal}
\usepackage[natbib=true,style=authoryear,backend=bibtex,useprefix=true]{biblatex}
\addbibresource{reading.bib}
\usepackage{graphicx,color}

\title{RL}
\author{Reuben Brasher}
\date{\today}

\begin{document}

\frame{\titlepage}

\section[Outline]{}
\frame{\tableofcontents}

\section{Pictures of games}

\frame
{
   \frametitle{$k$-armed Bandit}
   
   %Refer to Fig. \ref{fig:narmed}.
   
   \begin{figure}[ht]
      \includegraphics[height=1.8in,keepaspectratio]{images/Slot\_machines\_at\_Wookey\_Hole\_Caves.JPG}
      \caption{From https://en.wikipedia.org/wiki/Slot\_machine} \label{fig:narmed}
   \end{figure}
}

\frame
{
   \frametitle{Markov Decision Process}
   
   %Refer to Fig. \ref{fig:mdp}.
   
   \begin{figure}[ht]
      \includegraphics[height=1.8in,keepaspectratio]{images/Reinforcement\_learning\_diagram.svg.png}
      \caption{From https://en.wikipedia.org/wiki/Reinforcement\_learning} \label{fig:mdp}
   \end{figure}
}

\section{\textit{k}-armed Bandits}

\frame
{
   \frametitle{Problem}

   Each round $t$ an agent may choose an action from $k$ possible. The agent
   receives a reward sampled from a distribution conditioned on the action.
   $$P(R_t|A_t)$$
   The objective of the game is to learn which action will give the highest
   expected reward.
}

\frame
{
   \frametitle{$q_*$}

   \begin{itemize}
      \item<1-> If only we knew the \textit{value} of each action

      $$q_*(a) = \mathbb{E} [R_t | A_t=a].$$

      \item<2-> We do not. We know $Q_t(a)$.

   \end{itemize}
}


\frame
{
   \frametitle{Explore vs. Exploit}

   \begin{itemize}
      \item<1-> Greedy strategy is always choose current best 
      $$\argmax_a Q_t(a)$$
      
      \item<2-> $\varepsilon$-Greedy strategy is to choose uniformly randomly
      probability $\varepsilon$, and to follow greedy strategy otherwise.
      
      \item<3-> Upper confidence bound strategy is to choose
      $$\argmax_a \sqbrak{Q_t(a) + c \sqrt{\frac{\ln t}{N_t(a)}}}$$
      
   \end{itemize}
}

\section{Reinforcement Learning}

\frame
{
   \frametitle{Finite Markov Decision Process}
   
   Agent and environment interact to produce a trajectory
   
   $$S_0,\,A_0,\,R_1,\,S_1,\,A_1,\,R_2,\,S_2,\,A_2,\,R_3,\dots$$
   
   State and reward depend on previous state and agent action
   
   $$p(s', r | s, a) = \Pr \prn{S_t=s', R_t=r | S_{t-1}=s,A_{t-a}=a}$$
}

\frame
{
   \frametitle{$G_t$}

   \begin{itemize}
      \item<1-> \textit{Return}
      $$G_t = R_{t+1} + R_{t+2} + R_{t+3} + \cdots + R_T$$

      \item<2-> \textit{Discounted return}
      $$G_t = R_{t+1} + \gamma R_{t+2} + \gamma^2 R_{t+3} + \cdots = \sum^\infty_{k=0} \gamma ^k R_{t+k+1}$$

   \end{itemize}
}

\frame
{
   \frametitle{Policy, value and action value function}
   
   Agent responds to environment by sampling from \textit{policy} $\pi(a|s)$.
   
   $Value$ of state $s$ is
   
   $$v_\pi (s) = \mathbb{E} [G_t|S_t = s]$$
   
   Action value function is
   $$q_\pi(s, a) = \mathbb{E} [G_t|S_t = s, A_t = a]$$ 
}

\frame
{
   \frametitle{If only we knew\dots}
  
   The \textit{optimal policy} $\pi_*$ or
   
   the \textit{optimal value} of state $s$
   
   $$v_* (s) = \max_\pi v_\pi(a)$$
   
   or the \textit{optimal action value function}
   
   $$q_*(s, a) = \max_\pi q_\pi(s, a)$$
   
   Note that
   $$q_*(s, a) = \mathbb{E} \sqbrak{R_{t+1} + \gamma v_* \prn{S_{t+1}} | S_t = s, A_t=a}$$ 
}

\frame
{
   \frametitle{Bellman optimality equations}
   
   \begin{itemize}
      \item<1-> Bellman equation is 
      $$v_\pi(s) = \sum_a \pi(a|s) \sum_{s',r} p(s',r| s, a) \sqbrak{r + \gamma v_\pi(s')}$$
      
      \item<2-> Bellman optimality equation for value is
      $$v_*(s) = \max_a \sum_{s', r} p(s', r| s, a) [r + \gamma v_*(s')]$$
      
      \item<3-> Bellman optimality equation for action value is
      $$q_*(s,a) = \sum_{s', r} p(s', r| s, a) [r +  \gamma \max_{a'} q_*(s', a')]$$
      
   \end{itemize}
}

\frame
{
   \frametitle{Finding optimal policy}
   
   \begin{itemize}
      \item<1-> By policy iteration (pg 121)
      $$\pi_0 \to v_{\pi_0} \to \pi_1 \to v_{\pi_1} \to \cdots \to \pi_* \to v_*$$
      
      \item<2-> By computing $q_*$
      
      \item<3-> By computing $v_*$
      
   \end{itemize}
}

\frame
{
   \frametitle{Temporal Difference}
   Iteratively update $V$ to estimate $v_\pi$.   
   
   Update $V$ by
   $$V(S_t) \leftarrow V(S_t) + \alpha \sqbrak{G_t - V(S_t)}$$
   
   Why wait though?
   $$V(S_t) \leftarrow V(S_t) + \alpha \sqbrak{R_{t+1} + \gamma V(S_{t+1}) - V(S_t)}$$
}

\frame
{
   \frametitle{SARSA}
   Iteratively update $Q$ to estimate $q_*$. (pg 151)   
   
   Update $Q$ by
   $$Q(S_t,A_t) \leftarrow Q(S_t, A_t) + \alpha \sqbrak{R_{t+1} + \gamma Q(S_{t+1}, A_{t+1}) - Q(S_t,A_t)}$$
}

\frame
{
   \frametitle{Q-learning}
   Iteratively update $Q$ to estimate $q_*$. (pg 153)   
   
   Update $Q$ by
   $$Q(S_t,A_t) \leftarrow Q(S_t, A_t) + \alpha \sqbrak{R_{t+1} + \gamma \max_a Q(S_{t+1}, a) - Q(S_t,A_t)}$$
}

\frame
{
   \frametitle{Policy Gradient}
   Parameterize policy $\pi$ by $\theta$, for example
   $$\pi(a|s, \theta) = \frac{e^{h(s, a, \theta)}}{\sum_b e^{h(s, b, \theta)}}$$
   where $h(s, a, \theta)$ is ANN or similar.

   Then $\pi$ has a gradient with respect to $\theta$, and $v_\pi$ can be 
   improved by gradient ascent.
}

\frame
{
   \frametitle{Actor critic}
   Parameterize policy $\pi$ by $\theta$, and estimate $v$ by $w$
   $$\pi(a|s, \theta) = \frac{e^{h(s, a, \theta)}}{\sum_b e^{h(s, b, \theta)}}$$
   where $h(s, a, \theta)$ is ANN or and so is $v(s, w)$.

   Then $\pi$ has a gradient with respect to $\theta$, and $v$ with respect to
   $w$. Improve $v$ and $\pi$ in turn by applying gradient updates.
}

\frame
{
   \frametitle{Futher reading}
   \cite{sutton1998reinforcement}
   
   \cite{narasimhan2015language}
   
   \cite{li2016deep}
   
   \cite{mnih2015human}
   
   \cite{zhu2016target}
}

\begin{frame}[t,allowframebreaks]
\frametitle{References}
\printbibliography
\end{frame}

\end{document}
