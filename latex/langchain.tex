\documentclass{beamer}

% Set a theme for the presentation
\usetheme{Madrid}

% Title and author information
\title{6 Best Features of LangChain}
\subtitle{Powerful Tools for Language Model Applications}
\author{Reuben Brasher}
\date{\today}

\begin{document}

% Title slide
\begin{frame}
    \titlepage
\end{frame}

% Outline slide
\begin{frame}{Outline}
    \tableofcontents
\end{frame}

% Section 1: Templates
\section{Templates}

\begin{frame}{Templates: What they are}
    \begin{itemize}
        \item A template in LangChain is a predefined structure that is used to generate prompts for language models.
        \item It allows dynamic input by inserting variables into a standard format.
        \item Templates help you reuse prompt structures across different inputs.
    \end{itemize}
\end{frame}

\begin{frame}{Templates: Why they are valuable}
    \begin{itemize}
        \item \textbf{Efficiency}: You can quickly generate multiple prompts without needing to manually rewrite them.
        \item \textbf{Consistency}: Ensures uniformity in how you structure interactions with the model.
        \item \textbf{Flexibility}: You can easily adapt the template to different tasks (e.g., Q\&A, summarization, translations).
    \end{itemize}
\end{frame}

\begin{frame}[fragile]{Templates: Code Example}
    \begin{verbatim}
    from langchain.prompts import PromptTemplate

    template = "Summarize the following text: {text}"
    prompt_template = PromptTemplate(input_variables=
                                     ["text"],
                                     template=template)
    filled_prompt = prompt_template.format(text=
                              "LangChain is powerful.")
    print(filled_prompt)
    \end{verbatim}
\end{frame}

% Section 2: Memory
\section{Memory}

\begin{frame}{Memory: What it is}
    \begin{itemize}
        \item Memory in LangChain allows the model to maintain context across interactions.
        \item It can store conversation history, track user inputs, and recall entities in multi-turn conversations.
        \item Supports several types: conversation buffer, conversation summarization, and entity memory.
    \end{itemize}
\end{frame}

\begin{frame}{Memory: Why it is valuable}
    \begin{itemize}
        \item \textbf{Context Retention}: Ensures the language model can understand and refer back to previous interactions.
        \item \textbf{Improved Coherence}: Memory helps the model generate more coherent responses in multi-turn dialogues.
        \item \textbf{Personalization}: Can retain user preferences across conversations.
    \end{itemize}
\end{frame}

\begin{frame}[fragile]{Memory: Code Example}
    \begin{verbatim}
    from langchain.memory import ConversationBufferMemory

    memory = ConversationBufferMemory()
    memory.save_context({"input": "What is AI?"},
                        {"output":
                        "AI is love."})
    print(memory.load_memory_variables({}))
    \end{verbatim}
\end{frame}

% Section 3: Retrieval
\section{Retrieval}

\begin{frame}{Retrieval: What it is}
    \begin{itemize}
        \item Retrieval allows language models to query external data sources or document stores.
        \item It enables the model to fetch relevant documents or information before generating a response.
        \item Used in conjunction with Retrieval-Augmented Generation (RAG).
    \end{itemize}
\end{frame}

\begin{frame}{Retrieval: Why it is valuable}
    \begin{itemize}
        \item \textbf{Extended Knowledge}: Allows language models to go beyond their built-in knowledge by accessing external documents.
        \item \textbf{Real-Time Information}: Enables the retrieval of up-to-date or domain-specific information.
        \item \textbf{Increased Accuracy}: Improves the accuracy of model outputs by grounding them in relevant documents.
    \end{itemize}
\end{frame}

\begin{frame}[fragile]{Retrieval: Code Example}
    \begin{verbatim}
    retriever = db.as_retriever(search_type=
                                "similarity",
                                search_kwargs=
                                {"k": 5})
    result = retriever.get_relevant_documents(
        "What is machine learning?")
    print(result)
    \end{verbatim}
\end{frame}

% Section 4: Vector Data Stores
\section{Vector Data Stores}

\begin{frame}{Vector Data Stores: What it is}
    \begin{itemize}
        \item Vector data stores (like FAISS, Pinecone) store text embeddings as vectors for fast similarity search.
        \item They enable efficient searching and retrieval of relevant text chunks based on semantic similarity.
    \end{itemize}
\end{frame}

\begin{frame}{Vector Data Stores: Why it is valuable}
    \begin{itemize}
        \item \textbf{Scalability}: Vector stores allow retrieval of relevant information from large datasets.
        \item \textbf{Efficient Retrieval}: Enables fast searches across document embeddings for similarity.
        \item \textbf{Contextual Search}: Searches are based on semantic meaning, providing better results than keyword searches.
    \end{itemize}
\end{frame}

\begin{frame}[fragile]{Vector Data Stores: Code Example}
    \begin{verbatim}
    from langchain.vectorstores import FAISS
    from langchain.embeddings import OpenAIEmbeddings

    docs = ["This is document 1", "This is document 2"]
    embeddings = OpenAIEmbeddings()
    db = FAISS.from_documents(docs, embeddings)
    \end{verbatim}
\end{frame}

% Section 5: Document Readers
\section{Document Readers}

\begin{frame}{Document Readers: What it is}
    \begin{itemize}
        \item LangChain supports various document readers that allow you to load, parse, and extract text from documents.
        \item It supports formats like PDF, Word, and web pages.
    \end{itemize}
\end{frame}

\begin{frame}{Document Readers: Why it is valuable}
    \begin{itemize}
        \item \textbf{Multi-format Support}: Can handle multiple document formats (PDFs, Word, web content), making it flexible.
        \item \textbf{Automated Parsing}: Automatically extracts and cleans text from various document types.
        \item \textbf{Seamless Integration}: Works well with retrieval and embedding processes.
    \end{itemize}
\end{frame}

\begin{frame}[fragile]{Document Readers: Code Example}
    \begin{verbatim}
    from langchain.document_loaders import PyPDFLoader

    loader = PyPDFLoader("path/to/document.pdf")
    documents = loader.load()
    print(documents)
    \end{verbatim}
\end{frame}

% Section 6: Chunking Strategies
\section{Chunking Strategies}

\begin{frame}{Chunking Strategies: What it is}
    \begin{itemize}
        \item Chunking strategies allow LangChain to split large documents into smaller, manageable parts.
        \item This is necessary because language models have token limits and can't process very large documents at once.
    \end{itemize}
\end{frame}

\begin{frame}{Chunking Strategies: Why it is valuable}
    \begin{itemize}
        \item \textbf{Token Limits}: Helps in processing large documents by breaking them into smaller pieces.
        \item \textbf{Improved Search}: Allows retrieval systems to search relevant chunks of documents.
        \item \textbf{Efficiency}: Reduces the load on the language model by limiting the amount of text it needs to process at once.
    \end{itemize}
\end{frame}

\begin{frame}[fragile]{Chunking Strategies: Code Example}
    \begin{verbatim}
    from langchain.text_splitter import (
         RecursiveCharacterTextSplitter as Splitter
    )
    text_splitter = Splitter(chunk_size=1000,
                             chunk_overlap=150)
    chunks = text_splitter.split_documents(documents)
    print(chunks)
    \end{verbatim}
\end{frame}

% End of the presentation
\end{document}
