\documentclass{beamer}

\input{MathMacros}

% packages
\usepackage{beamerthemesplit}
\usepackage{enumerate}
\usepackage{amssymb}
\usepackage[all,cmtip]{xy}
\usepackage[mathscr]{eucal}
\usepackage[natbib=true,style=authoryear,backend=bibtex,useprefix=true]{biblatex}
\addbibresource{reading.bib}
\usepackage{graphicx,color}

\title{Testing in Software Development}
\author{Reuben Brasher}
\date{\today}

\begin{document}

\frame{\titlepage}

\section[Outline]{}
\frame{\tableofcontents}

\section{Diffusion models}

\frame
{
   \frametitle{Why?}

    Why test? Seriously.
}

\frame
{
   \frametitle{Automated vs. Manual}

    Code accumulates.

    Tests must accumulate too.
}

\frame
{
   \frametitle{Collaborative vs. Individual}

    You are not the only person in the world.

    Your friends love you, but they do not have time to read all of your code and docs.
}

\frame
{
   \frametitle{Reality}

    If it is not tested, then it is broken.

    If it is not tested, then it does not exist.

    If it is documented, then the documentation lies.

    If you fix a bug without adding a test, you did not fix the bug.
}


\frame
{
   \frametitle{Levels of testing}

    Unit testing

    Integration testing

    Functional testing

    Regression testing

    And others
}

\frame
{
   \frametitle{Unit testing}

    Test units of code, typically at the method or function level.
}


\frame
{
   \frametitle{Integration Testing}

    Test the interactions between different components of a system to ensure they work together.
}

\frame
{
   \frametitle{Functional Testing}

    Test that code functions according to specifications. Functional tests verify that code meets standards and
    implements documented behavior.
}

\frame
{
   \frametitle{Regression Testing}

    Test that behavior of code does not change when code changes.
}



\begin{frame}[t,allowframebreaks]
\frametitle{References}
\printbibliography
\end{frame}

\end{document}
