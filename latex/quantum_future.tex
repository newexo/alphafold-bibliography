\documentclass{beamer}

\input{MathMacros}

% packages
\usepackage{beamerthemesplit}
\usepackage{enumerate}
\usepackage{amssymb}
\usepackage[all,cmtip]{xy}
\usepackage[mathscr]{eucal}
\usepackage[natbib=true,style=authoryear,backend=bibtex,useprefix=true]{biblatex}
\addbibresource{reading.bib}
\usepackage{graphicx,color}
\usepackage{braket}

\title{How Will the Quantum Future Become the Quantum Now?}
\author{Reuben Brasher}
\date{\today}

\begin{document}

\frame{\titlepage}

\section[Outline]{}
\frame{\tableofcontents}

\section{Quantum is Now}

\frame
{
   \frametitle{What to we want?}

   $$U(\theta) \ket{\psi} > = \ket{\phi},$$

   where $U(\theta)$ is a unitary transform parameterized (perfectly) by $\theta$ and $\psi$ encodes arbitrary classical
   data.
}

\frame
{
   \frametitle{If we cannot have that?}

   $$U(\theta_\psi) \ket{0} = \ket{\psi},$$

   where $U(\theta_\psi)$ is a unitary transform parameterized (perfectly) by $\theta_\psi$ depending on $\psi$. Hence

   $$U(\theta) U(\theta_\psi) \ket{0} = \ket{\phi}.$$
}

\frame
{
   \frametitle{If we cannot have that?}

   $$U(\theta) \ket{0} = \ket{something},$$

   where $U(\theta)$ kind of depends on classical data.
}

\frame
{
   \frametitle{What is a computer?}

   A computer is an isolated system for performing repeatable experiments controlled by user input.
}

\frame
{
   \frametitle{What is a quantum computer?}

   A computer is an isolated system for performing repeatable experiments controlled by user input.
}

\frame
{
   \frametitle{What is the Quantum Now?}

   \begin{itemize}
      \item<1-> A practical reality in which we understand that the math of quantum mechanics can define probability distributions
   which can be sampled to solve real problems.

      \item<2-> Computation is distributed. It may be quantum, and it may be classical. It may be CPU, and it may be
   GPU.

      \item<3-> Computation is layered in classical and quantum layers.

      \item<4-> Both classical and quantum computations are parameterized and the parameterized and the parameters of
   the computations are learn.

   \end{itemize}

}

\frame
{
   \frametitle{What is the Quantum Now?}

   Quantum computation is ML.
}

\begin{frame}[t,allowframebreaks]
\frametitle{References}
\printbibliography
\end{frame}

\end{document}
